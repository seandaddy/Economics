\section{Introduction to the Neoclassical Model of Trade}

\subsection{Introduction}

The neoclassical model assumes that there are many production processes available to produce a good. These processes represent a production set in which various combinations of factors of production are feasible. Firms maximize profits by choosing optimal production processes from the feasible production set. To apply calculus, it is often assumed that a production function can be expressed as a continuously differentiable function of inputs. Thus, whenever the factor prices change, the firms can substitute one factor for another in a smooth fashion. In the competitive case, the neoclassical model also assumes that both firms and factors of production are price takers.

In this chapter, we study the neoclassical production structure in a two-good and two-factor model. We show that, from a general equilibrium perspective, a competitive economy is Pareto efficient in the autarkic state and that the presence of a monopoly causes inefciency in an econ- omy. We then study the structure of the competitive world economy by developing the techniques of offer curves and trade indifference curves. These techniques are used to prove the results that free trade is Pareto efficient and that gains from trade exist.

\subsection{The Efficiency of a Competitive Economy}

Consider a simple two-good, two-factor model. Let the two sectors' production function be 

\begin{aligned}
  S_j=F^j\left(K_j, L_j\right), j=1,2,
\end{aligned}

where $K_j$ and $L_j$ are capital and labor employed in sector $j$. Let $R_j$ and $C_j$ be the total revenue and total cost in sector $j$, each of which is a function of output $S_j$, and $\Pi_j$ be its profit:

\begin{aligned}
\Pi_j & =R_j\left(S_j\right)-C_j\left(S_j\right) \\
& =R_j\left(F^j\left(K_j, L_j\right)\right)-\left(r K_j+w L_j\right)=\Pi_j\left(K_j, L_j\right)
\end{aligned}

Maximization of $\Pi_j$ with respect to $S_j$ yields the equality condition between marginal revenue and marginal cost:

\begin{aligned}
  R_j^{\prime}=C_j^{\prime}
\end{aligned}

Equivalently, Maximization of $\Pi_j$ with respect to $K_j$ and $L_j$ yields

\begin{aligned}
  r = R_j^{\prime} F_1^j , w=R_j^{\prime} F_2^j ,
\end{aligned}

where $F_1^j \equiv \partial F^j / \partial K_j$ and $F_2^j \equiv \partial F^j / \partial L_j$ are the marginal physical products of capital and labor, respectively. Thus, profit maximization equates factor prices to their respective values of marginal products. If the industry is perfectly competitive, then $R_j = \bar{p_j}S_j$ and the marginal revenue $R_j^{\prime} = \bar{p_j}$, where $\bar{p_j}$ is a constant to the firm. However, if the industry is not perfectly competitive, firms are not price takers and $p_j$ is affected by their actions.

Full employment conditions are:

\begin{aligned}
  K_1 + K_2 &= K,//
  L_1 + L_2 &= L.
\end{aligned}

With full employment of both factors and the given technologies embodied in the production functions, there is a maximum $S_2$ that can be produced for a given $S_1$ in the economy. Let this maximum $S_2$ locus (the production possibility frontier, PPF) be

\begin{aligned}
S_2 = f(S_1).
\end{aligned}

The trade-off between the two output is the opportunity cost of one good in terms of the other. Let the marginal rate of transformation(MRT) be defined as $MRT \equiv -dS_2 / dS_1$. From (1), we obtain $dS_j = F_1^j dK_j + F_2^j dL_j$. Using (6), we have $dK_1 = -dK_2$ and $dL_1 = -dL_2$. It follows that

\begin{aligned}
  MRT = \frac{C_1^{\prime}}{C_2^{\prime}}.
\end{aligned}

If perfect competition prevails in each sector, we have

\begin{aligned}
  \frac{C_1^{\prime}}{C_2^{\prime}} =  
  \frac{R_1^{\prime}}{R_2^{\prime}} =  
  \frac{p_1}{p_2}.
\end{aligned}

Consumers maximize the utility function $U(D_1, D_2)$, subject to the budget constraint 

\begin{aligned}
  MRT = \frac{C_1^{\prime}}{C_2^{\prime}} =  
  \frac{p_1}{p_2} = MRS.
\end{aligned}

These are the efficiency conditions of a competitive equilibrium.

An important implication of the efficiency condition is that GDP is maximized in a competitive economy. Let the GDP be 

\begin{aligned}
  Y = p_1 S_1 + p_2 S_2 .
\end{aligned}

Since $MRT \equiv - \frac{dS_2}{dS_1} = \frac{p_1}{p_2}$ by (10), we have $p_1 dS_1 + p_2 dS_2 = 0.$ This implies that at a given $(p_1, p_2)$, we have $dY = p_1 dS_1 + p_2 dS_2 = 0,$ which is the necessary condition for a maximum GDP. Since each agent in the economy is assumed to care only for his or her own interest, no one is trying to maximize GDP. Nonetheless, the competitive economy's "invisible hand" maximizes the value of aggregate output.

\subsection{Monopoly in Sector 1}

If an economy has distortions that undermine perfect competition, efficiency is lost. For example, if sector 1 is monopolized by a firm and sector 2 is perfectly competitive, then $C_1^{\prime} = R_1^{\prime}<p_1$, and $C_2^{\prime}=R_2^{\prime}=p_2$. As a result, we have

\begin{aligned}
  MRT = \frac{C_1^{\prime}}{C_2^{\prime}} < \frac{p_1}{p_2} = MRS.
\end{aligned}

Fig.1 shows that the autarkic equilibrium is at point $C$ under perfect competition in both sectors and at point $M$ if sector 1 is monopolized. At point $C$, the efficiency conditions in (10) are satisfied so that the relative price $p_C$ is equal to the slope of the PPF; but at point $M$, the relative price $p_M$ is steeper than the slope of the PPF, as shown in (12). The existence of a monopoly results in lower social welfare ($U_M < U_C$). From a societal perspective, a monopoly in sector 1 causes underproduction of good 1 and overproduction of good 2, providing a fundamental economic basis for antitrust policies.

\subsection{The Box Diagram and the Production Efficiency Locus}

The construction of the production box diagram, as shown in Fig.2, involves isoquant curves. An isoquant curve depicts the various combinations of inputs that yield the same level of output. The trade-off between the two inputs, holding $S_j$ constant, is the marginal rate of technical substitution in sector $j, MRTS_j$, which is defined as $-d K_j / d L_j \mid_{\bar{s}_j}$. Total differentiation of $S_j = F^j (K_j, L_j)$ yields $0 = dS_j = F_1^j dK_j + F_2^j dL_j$, where $F_1^j = \partial{F^j}/\partial{K_j}$ and $F_2^j = \partial{F^j}/\partial{L_j}$. This implies

\begin{aligned}
  MRTS_j = - \frac{dK_j}{dL_j} \mid_{S_j} = \frac{F_2^j}{F_1^j}.
\end{aliigned}

The MRTS is the absolute value of the slope of an isoquant.

To produce a given output level at $\bar{S_j}$, a firm will try to minimize its cost, $rK_j + wL_j$. Given $r$ and $w$, the necessary condition is $rdK_j + wdL_j = 0$, which implies $-dK_j / dL_j \mid_{S_j} = w/r$. It follows that

\begin{aligned}
  MRTS_1 =  \frac{F_2^1}{F_1^1} = \frac{w}{r} = \frac{F_2^2}{F_1^2} = MRTS_2 .
\end{aligned}

If the two sectors face the same wage/rental ratio, $w/r$, they must have the same marginal rate of technical substitution; namely, their isoquant curves must have the same slope at their production points.

The box diagram shown in Fig.2 is constructed from the two sectors' isoquant maps. The horizontal length of the box is the total labor endowment $L$ in the economy, and the vertical length is the capital endowment $K$. Any point in the box shows the employments of $K_j$ and $L_j$ in each sector.

Assume that the technologies in both sectors are characterized by constant returns to scale(CRS). A CRS production function implies $tS_j = F^j (tK_j, tL_j)$ for any $t \geq 0$. Thus, for example, doubling all of the inputs in a sector($t=2$) doubles the output. A CRS production function is also called linear homogeneous, or homogeneous of degree one(h.o.d.1). Given that $F^j(K_j, L_j)$ is h.o.d.1, its partial derivatives are homogeneous of degree zero, and they can be expressed as function of the sectoral capital-labor ratio, $k_j (\equiv K_j / L_j)$. $k_j$ is also called the capital intensity in sector $j$. Under CRS, the marginal physical product of a factor of production is a function of the capital intensity in that sector. It follows the $MRTS_j$ is also a function of $k_j$. As $k_j$ increases, $F_2^j$ increases and $F_1^j$ decreases. Thus, the $MRTS_j$ is a monotonically increasing function of $k_j$. Therefore, for a given $w/r$, the cost minimizing $k_j$ is unique, and the optimal $k_j$ is an increasing function of $w/r$.

From the box diagram in Fig.2, the efficiency locus $O_1ABO_2$ is the locus that satisfies the efficiency condition $MRTS_1 = MRTS_2$. This locus is called "the contract curve," which is the locus of all tangent points of the two isoquant maps. Other points not on the contract curve do not satisfy the efficiency condition, although the economy is capable of producing at those points with both factors of production fully employed. As one moves from $O_1$ to $A$ and $B$, $S_1$ increases and $S_2$ decreases. Points $C, D,$ and $A$ all yield the same $S_1$, and points $C, D,$ and $B$ all yield the same $S_2$. As one moves from $C$ or $D$ toward the efficiency segment $AB$ both outputs will increase. Thus, any point off the contract curve is inefficient; but if on the curve, any move away from it will necessarily decrease one sector's output and possibly both. The contract curve is the set of Pareto efficient allocations.

Note that at point $D$, for example, the two sectors' marginal rates of technical substitution are not the same: $MRTS_1 > MRTS_2$. If point $D$ is the actual production point in the economy, it can be due to both sectors not facing the same wage/rental ratio. Suppose that the distortion is due to the labor market, in which sector 1 is unionized while sector 2 is not so that $w1 > w2$. In this case, we have $MRTS_1 = w_1/r > w_2/r = MRTS_2$. This illustrates a case in which a distorted factor market causes the economy to stray from its efficiency locus.

Note also that at point $A$. as at any point on the contract curve shown in Fig.2, sector 1's capital intensity exceeds sector 2's ($k_1 > k_2$). This is due to the assumption that sector 1 is more capital intensive than sector 2. If this assumption were reversed, the efficiency locus would lie below the diagonal line. In the special case where the two sectors have identical factor intensities ($k_1 = k_2$), the efficiency locus is the diagonal line itself. Under CRS, if a point on the diagonal line is on the efficiency locus, then the whole diagonal line itself is the efficiency locus.

\subsubsection{The Production Possibility Frontier}

The production possibility frontier, or PPF, is the locus of the maximum $S_2$ for any given $S_1$ with given technology and factor endowments. The intersection of the curve and the $S_1$-axis corresponds to the origin $O_2$ in the box diagram and indicates the highest attainable level of $S_1$. Similarly, the intersection with the $S_2$-axis corresponds to $O_1$ in the box diagram. The PPF is mapped from the efficiency locus in the box diagram to the output space. Points outside the PPF are infeasible; points within it are feasible but inefficient. Points $A$ and $B$ are both on the contract curve and hence are both on the PPF. Points $C$ and $D$ are not on the contract curve and therefore must lie in the interior of the production possibility set. Since $C$ and $D$ have the same output configuration, they are mapped to the same point in the production possibility set.
The diagonal line in the box diagram is mapped to the dotted line joining the two end points of the PPF. The PPF, therefore, must be a concave curve above the dotted line. Only when the two sectors have identical factor intensities ($k_1 = k_2$) will the PPF become the dotted line itself.6 In this special case, the marginal rate of transformation is constant for any output configuration. The PPF normally has a bowed-out shape. Thus, the marginal rate of transformation, MRT ($\equiv -dS_2 / dS_1$), which measures the opportunity cost of producing good 1 in terms of good 2, is increasing in $S_1$. This demonstrates increasing opportunity costs or the law of diminishing returns.

\subsection{The Trade Indifference Curves and the Offer Curve}

Consider a given community indifference curve (CIC), like the one shown in the second quadrant of Fig.4, which is defined by a given $\bar{U} = U (D_1, D_2)$. The same quadrant also depicts $(S_1, S_2)$ with a PPF going through point $A$. The autarkic equilibrium occurs at point $A$ where $MRS = MRT$. The common tangent line is the consumers' budget line, $p_1 D_1 + p_2 D_2 = Y$, which is also the gross national product line $p_1 S_1 + p_2 S_2 = Y$, both of which have the absolute slope $p$.

To attain the same CIC but to have a different consumption point, such as $B$, the nation must engage in foreign trade. Consider sliding the PPF along the given CIC while keeping the orientation of the axes unchanged. At the common tangent point $B$, the consumption vector is $OB$, but the production vector is $B^{\prime}B$; hence, the required trade vector is $OB^{\prime}$. Similarly, if the consumption point is at $C$, the required trade vector is $OC^{\prime}$. In this case, the nation must export $G2$ in exchange for $G1$. Thus, as the production possibility set slides along a given CIC, its origin traces out a trade indifference curve (TIC), as shown in Fig.4. If a different CIC is chosen, a new TIC will be generated. In the first quadrant, for a given $X$, the higher the TIC, the higher the national welfare.

The slope of the TIC at $B^{\prime}$ is the same as the common slope at $B$ of the CIC and the PPF. Thus, the slope of TIC, called the marginal rate of substitution in trade(MRST), is equal to $p$.

Fig.5 depicts a number of TICs. Under free trade, the balance-of-trade equilibrium requires that

\begin{aligned}
  M=p^* X
\end{aligned}

where $p^* = p_1^* / p_2^*$ is the terms of trade($=M/X$). Eq.(15) depicts a ray from the origin with the slope $p^*$. If $p^*$ increases, the home country's terms of trade are improved. Utility-maximizing trade thus occurs at the tangent point of the highest attainable TIC and the terms-of-trade line. The locus of the optimal trading points for all possible terms of trade is the offer curve (labeled $0O$), or the reciprocal demand curve. This is the approach used in Meade(1952) to derive the offer curve.

The offer curve is a "no-free-lunch" curve. In order to import, the nation must export. As one extra unit of $G1$ is exported, its marginal utility of $G1$ increases since less units are available for home consumption, and as imports increase, the marginal utility of $G2$ decreases. Thus, the opportunity costs of exporting successive units of $G1$ are increasing in terms of $G2$, and the offer curve in general bows out toward the country's export axis.

\subsection{Free Trade Equilibrium}

The determination of the equilibrium terms of trade and volumes of trade can now be obtained by the offer curve technique. Fig.6 shows the two offer curves, $0O$ and $0O^*$, intersect at T which is the equilibrium trading point. The equilibrium terms of trade ($p^t = p^{*t}$), are the slope of the $OT$ line. At $T$ $X = M^*$ and $M = X^*$. The community indifference curves for both $H$ and $F$ relevant for the free trade equilibrium are shown as $CIC^t$ and $CIC^{*t}$. For $H$, the point of tangency between its PPF and $CIC^t$ is $D^t$ which generates the domestic price line facing home consumers and producers; likewise, the point of tangency for $F$ is $D^{*t}$, which generates the $p^{*t}$ line facing foreign consumers and producers. At $T$, each country reaches its highest attainable trade indifference curve (not drawn), and the two trade indifference curves are tangent to each other with the common slope $p^t = p^{*t}$.

Fig.6 shows a great deal of information about each country's production, consumption, trade volumes, welfare level, and equilibrium terms of trade. It also shows the efficiency properties under free trade for the world economy:

\begin{aligned}
  MRS = MRT = MRST = p^t = p^{*t} = MRST^* = MRT^* = MRS^*. 
\end{aligned}

One can experiment with the gure to show that if any of the above equality conditions are violated, then at least one country will be worse off. In particular, any move away from T will reduce at least one country's welfare. Thus, free trade is Pareto efficient. Under free trade, all consumers in the world maximize their utilities within their budget constraints, all producers in the world maximize their profits, and the world GDP is also maximized.

\subsection{The Gains from Trade}

Fig.7 shows that if there is no trade, both nations are at the origin $0$, and each nation is at its autarkic level of the trade indifference curve: $TIC^0$ and $TIC^{*0}$. By moving from $0$ to $T$, each country attains a higher trade indifference curve: $TIC^t$ and $TIC^{*t}$. The gains from trade for both countries are clearly shown. The slope of the $TIC^0$ at the origin is Home's autarkic price ratio $p$, and the slope of the $TIC^{*t}$ at the origin is Foreign's autarkic price ratio $p^*$. Since $p < p^t = p^{*t} < p^*$, each country enjoys the benefit of specializing more in the good in which it has a comparative advantage. Unlike the classical Ricardian model, both countries are still likely producing both goods - as the presence of increasing opportunity costs in the neoclassical model prevents specialization from being complete.

The $KK$ locus in Fig.7 is the locus of all common tangent points of the two countries' trade indifference maps. Point $T$ lies in $KK$. If the world economy is not perfectly competitive, as in the case where each country has a state trading agency to conduct foreign trade, then the final outcome of trade may not be settled at point $T$. Since each country must be at least no worse off than in the autarkic case to engage in voluntary trade, the potential bargaining set is the region bounded by the two no-trade curves - $TIC^0$ and $TIC^{*0}$ - that are their respective reservation utilities. A point in the bargaining set but not on the $KK$ curve, however, will not be an efficient outcome because there can be a move toward the $KK$ curve that will make at least one country better off without reducing the other's utility. Once on KK; any further move will make at least one country worse off and may even cause both to be worse off. The $KK$ locus is called the contract curve which is the set of Pareto efficient points.

From Home's perspective, the best point without causing Foreign to withdraw from bargaining is $A$, while from Foreign's perspective, $B$ is the best point. So the bargaining outcome will lie in the segment $AB$, which is the "core" of the economy. Any free trade equilibrium, therefore, must be in the core. As there are more agents in the market, each agent's power is diminished, and the set of potential outcomes from bargaining will shrink; i.e., the core will diminish in size. If no agent has bargaining power, then the core approaches the set of competitive equilibria. It should be noted that the competitive equilibrium may not be unique. One can imagine that with various patterns of CIC's and the PPF, the offer curves may have very strange shapes and may intersect with each other at multiple points.

\subsection{Exercises}

\begin{enumerate}
    \item If Sector 1 has a consumption tax of $t_{ie}$ per unit of consumption, does the quality between MRT and MRS still hold? If not, how are they related? [Hint: Let $p_{1e}$ and $p_{1p}$ be the prices facing sonsumers and producers. Then $p_{ie} = (1+t_{1e})p_{ip}$.]
    \item If both sectors are given the same percentage production subsidy based on their respective marginal costs, does the equality between MRT and MRS still hold? Explain.
    \item Assume that both sectors have a monopoly and both monopolies have the same measuere of monopoly power, \[
     \frac{p_1 - C_1^{\prime}}{p_1} = \frac{p_2 - C_2^{\prime}}{p_2} .
    \] 
    Does the equality between MRT and MRS still hold? Prove your assertion. Show the equilibrium production and consumption point on PPF and compare this case with the perfectly competitive case. [Hint: Does the above equation imply $\frac{C_1^{\prime}}{C_2^{\prime}} = \frac{p_1}{p_2}$?]
  \item Consider the production box diagram in (2). Answer the following questions
    \begin{enumerate}
      \item At A, is the economy fully employed? How about at C?
      \item Compare C and D. Do they have the same output configuration?
      \item At A and B or any other points on the contract curve, do the two sectors face the same wage/rental rate ratio($w/r$)?
      \item If Sector 2's labor is unionized but Sector 1's is not, which point depicts this situation: C or D? 
    \end{enumerate}
  \item Assume that Home and Foreign have identical technologies(i.e. the same production function in each sector) and that sector 1 is capital intensive ($k_1 > k_2$ for any $w/r$). If $K>K^*$ and $L=L^*$, draw the PPF's for both conutries and compare their shapes. 
  \item Why is it the case that the PPF is above the straight line joining the two ends of the PPF?
  \item Is it true that any point in the interior of the production possibility set must mean unemployment of some factors so that the economy has some idle resources? Explain.
  \item Why is free trade Pareto efficient?
\end{enumerate}
