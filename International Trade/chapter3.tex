\section{Simple Pure Exchange Modles}

\subsection{Introduction}

The price of a good or service observed in a national market, without foreign trade, is determined by demand and supply in that nation. On the supply side, nations generally have different natural endowments, different factors of production, and different technologies relevant to the production of goods and services. These differences result in variations in production costs. On the demand side, people have different incomes and tastes, which generate unique patterns of demand among nations. Thus, the autarkic prices observed among nations tend to be different. Once foreign trade is opened up, goods and services move between nations. Questions that come to mind are: Which country exports which good? What role do autarkic prices play in determining the direction of trade? What are the equilibrium terms of trade? Are there gains from trade? These are some of the basic questions to be answered in the study of international economics.

In this chapter, we will consider some simple exchange models in order to illustrate the deter- minants of trade. It will be shown that it is the difference in relative autarkic prices, not money prices, that causes trade to take place. We will also examine the determination of the equilibrium terms of trade and gains from trade with the help of a simple exchange model, while keeping the supply of goods fixed.

\subsection{The Determinants of Barter Trades}

To illustrate the basic determinants of trade, we consider a very simple pure exchange model in which there are two countries, called home (H) and foreign (F), and two goods, good 1 (G1) and good 2 (G2). Goods are already produced so there are no new production activities in either country. Before trade is opened up, each good has an autarkic price in each country. Let the home country's variables be represented without an asterisk and foreign variables with an asterisk. Let pj be the money price of good j in H ($j = 1, 2$), measured, say, in U.S. dollars and $p_j^{\prime}$ be the money price of good j in F ($j = 1, 2$), measured in British pounds. These are the money (nominal) prices observed in the autarkic (no foreign trade) equilibrium. Assume that there are no impediments to trade; namely, there are no transport costs or other trading costs. The markets are competitive, and the buyers and sellers all face the same prices. How is the trade pattern determined? Is it determined by these nominal prices?

To answer both questions, we rst consider a barter exchange economy in which one good is directly exchanged for the other. Since there are no trading costs, if the barter exchange ratios between the two goods in the two countries are different, gainful arbitrage activities will take place and trade will occur. A barter exchange ratio is nothing but the nominal price ratio. Let $p =p_1 / p_2$ which is called "the relative price of G1 (in terms of G2)" in H. It measures p units of G2 that are exchanged for one unit of G1. For example, if $p_1 = 10$ and $p_2 = 5$, then $p = 2$, which means one unit of G1 can be exchanged for two units of G2. Formally, $p_1 / p_2$ has the dimension of
$ \left( \frac{\$10}{1G1} / \frac{\$5}{1G2} \right)  =  2G2/1G1 $. The \$ element is canceled out in the ratio $p_1 / p_2$. Similarly, the relative price of G2 (in terms of G1) in H is $1/p = p_2 / p_1$. The foreign autarkic relative price of G1 in terms of G2 is $p^* = p_1^* / p_2^*$.

The following proposition shows that the determinants of trade are not the nominal autarkic prices but the relative autarkic prices.

Proposition 1 Barter trade between a pair of goods can gainfully take place if the autarkic relative prices between the pair of goods in the two countries are different. A country will export the good whose relative autarkic price is cheaper than the other's, and both countries can gain from the trade.

Consider the following autarkic price matrix:

\begin{equation}
\left[\begin{array}{ll}
p_1 & p_1^* \\
p_2 & p_2^*
\end{array}\right]=\left[\begin{array}{ll}
1 & 40 \\
2 & 10
\end{array}\right].
\end{equation}

From this matrix, we know that one unit of G1 in H can be exchanged for $0.5$ units of G2 ($p = 1/2 = 0.5$), but in the foreign country, one unit of G1 can be exchanged for four units of G2 ($p^* = 40/10 = 4$). So if a merchant ships one unit of G1 from H to F, he can exchange that unit for four units of G2 in F, compared with only $0.5$ units of G2 in H. He will have a gain of $3.5$ units of G2. If this gain is more than enough to pay for the transportation and other transactions costs, he will gain from the trade and export G1.

Now think of F. Under autarky, a unit of G2 can be exchanged for $p^* = p_2* / p_1^* \equiv 1/p^* = 10/40 = 0.25$ units of G1. If a merchant ships one unit of G2 from F to H, he can exchange that unit for $p_2 / p_1 = 2/1 = 2$ units of G1. He will gain $1.75$ units of G1. Thus, F will export G2. Unequal relative prices create the opportunity for profitable arbitrage, and since both countries gain from this transaction, trade emerges voluntarily.

The above example illustrates a fundamental principle: if, under autarky, $p_1 / p_2 < p_1^* / p_2^*$, then H will export G1, and F will export G2. It is easy to verify that if $p_1 / p_2 > p_1^* / p_2^*$, then the opposite trade pattern occurs. Thus, whenever p $p_1 / p_2 \neq p_1^* / p_2^*$, trade will take place. Furthermore, both countries gain from trade.

Note that the relative prices are independent of currency units used in each country. If H is the U.S., F is the U.K., and $£ 1=\$ 1$ then the U.K.'s autarkic money prices in terms of dollars are both higher than those in the U.S. If trade was thought to be determined by comparing the money prices, then the U.K. would not have been able to export any good. It is the autarkic relative prices that matter in the determination of the pattern of barter trade, not the nominal money prices. Since the currency exchange rate has no bearing on relative prices, it is not a factor in determining the pattern of trade in the barter exchange model.

\subsection{A Simple Model of Trade with Fixed Initial Endowments}


